\documentclass[ngerman,a4paper,abstracton,open=right,twoside=false,toc=listofnumbered]{scrreprt}
\usepackage[utf8]{inputenc}
\usepackage[T1]{fontenc}
\usepackage{lmodern}
\usepackage{babel}
\usepackage{geometry}
\usepackage{graphicx}
\usepackage[hidelinks]{hyperref}
\geometry{a4paper, top=20mm, bottom=40mm, left=40mm, right=25mm, footskip=20mm}

\title{Leistungsvergleich von Clojure--Parsern}
\subtitle{Technischer Bericht zum Masterprojekt\
Technische Hochschule Mittelhessen}
\author{Daniel Kirsten, Markus Bader}
\date{\today}
\begin{document}

\maketitle
\newpage
\begin{abstract}

Im Rahmen des Masterprojekts im Wintersemester 2013/2014 haben die Autoren ein
Programm zum Anwenden von Logikfunktionen (z. B. Wahrheitstafeln,
Tseitin--Transformation, SAT--Solving) entworfen und mittels der
Programmiersprache Clojure implementiert. Unter anderem wird ein Parser zum
Umwandeln von logischen Formeln in eine Clojure--Datenstruktur benötigt.

Zu diesem Zweck wurde ein Parser mittels der Bilbiothek Instaparse erzeugt,
jedoch viel dessen signifkant längere Laufzeit gegenüber einem
JavaCC--Vergleichsparser auf. 

In diesem technischen Bericht, welcher zugleich die Abschlussdokumentation zum
Projekt darstellt, werden die Laufzeiten von fünf Parsern verglichen: Instaparse
und Kern, jeweils mit einer vollständigen und einer minimierten Grammatik, sowie
der JavaCC--Referenzparser.

\end{abstract}
\newpage
\tableofcontents
\newpage

\chapter{Einführung}
\section{Der MPA}
\section{Logical Workbench}
\section{Die Parser--Bibliotheken}
\subsection{Instaparse}
\subsection{Kern}
\subsection{JavaCC}

\chapter{Versuchsaufbau}
\section{Allgemeine Aspekte}
\section{Grammatiken}
\section{Formeln}
\section{Messungen}

\chapter{Bewertung}

\appendix
\listoffigures % Abbildungsverzeichnis

\end{document}
