\documentclass[ngerman,a4paper,abstracton,open=right,twoside=false,toc=listofnumbered]{scrreprt}
\usepackage[utf8]{inputenc}
\usepackage[T1]{fontenc}
\usepackage{lmodern}
\usepackage{babel}
\usepackage{geometry}
\usepackage{graphicx}
\usepackage[hidelinks]{hyperref}
\geometry{a4paper, top=20mm, bottom=40mm, left=40mm, right=25mm, footskip=20mm}

\title{Leistungsvergleich von Clojure--Parsern}
\subtitle{Technischer Bericht zum Masterprojekt\
Technische Hochschule Mittelhessen}
\author{Daniel Kirsten, Markus Bader}
\date{\today}
\begin{document}

\maketitle
\newpage
\begin{abstract}

Im Rahmen des Masterprojekts im Wintersemester 2013/2014 haben die Autoren ein
Programm zum Anwenden von Logikfunktionen (z. B. Wahrheitstafeln,
Tseitin--Transformation, SAT--Solving) entworfen und mittels der
Programmiersprache Clojure implementiert. Unter anderem wird ein Parser zum
Umwandeln von logischen Formeln in eine Clojure--Datenstruktur benötigt.

Zu diesem Zweck wurde ein Parser mittels der Bilbiothek Instaparse erzeugt,
jedoch viel dessen signifkant längere Laufzeit gegenüber einem
JavaCC--Vergleichsparser auf. 

In diesem technischen Bericht, welcher zugleich die Abschlussdokumentation zum
Projekt darstellt, werden die Laufzeiten von fünf Parsern verglichen: Instaparse
und Kern, jeweils mit einer vollständigen und einer minimierten Grammatik, sowie
der JavaCC--Referenzparser.

\end{abstract}
\newpage
\tableofcontents
\newpage

\chapter{Einführung}
\section{Der MPA}
\section{Logical Workbench}
\section{Die Parser--Bibliotheken}
\subsection{Instaparse}
\subsection{Kern}
\subsection{JavaCC}

\chapter{Versuchsaufbau}
\section{Allgemeine Aspekte}
\section{Grammatiken}
\section{Formeln}
\section{Messungen}

\chapter{Bewertung}

\appendix
\listoffigures % Abbildungsverzeichnis

\chapter{Messergebnisse}
\begin{table}[h]
	\rotatebox{90}{
		\begin{tabular}{|l|r|r|r|r|}
			\hline
			\textbf{USA} & \multicolumn{1}{l|}{\textbf{}} & \multicolumn{1}{l|}{\textbf{}} & \multicolumn{1}{l|}{\textbf{}} & \multicolumn{1}{l|}{\textbf{}} \\ \hline
			javacc  & 208 & 217 & 206 & 196 \\ \hline
			kern, slim   & 610 & 814 & 832 & 859 \\ \hline
			kern, full   & 673 & 665 & 670 & 719 \\ \hline
			instaparse, slim   & 595 & 593 & 584 & 607 \\ \hline
			instaparse, full   & 4094 & 3652 & 3649 & 3607 \\ \hline
			\textbf{4 Damen} & \multicolumn{1}{l|}{} & \multicolumn{1}{l|}{} & \multicolumn{1}{l|}{} & \multicolumn{1}{l|}{} \\ \hline
			javacc  & 10 & 9 & 10 & 8 \\ \hline
			kern, full   & 199 & 199 & 191 & 190 \\ \hline
			instaparse, full   & 118 & 110 & 107 & 110 \\ \hline
			\textbf{8 Damen} & \multicolumn{1}{l|}{} & \multicolumn{1}{l|}{} & \multicolumn{1}{l|}{} & \multicolumn{1}{l|}{} \\ \hline
			javacc  & 92 & 94 & 94 & 72 \\ \hline
			kern, slim   & 285 & 276 & 292 & 296 \\ \hline
			kern, full   & 645 & 624 & 567 & 649 \\ \hline
			instaparse, slim   & 298 & 369 & 345 & 291 \\ \hline
			instaparse, full   & 2351 & 2223 & 2249 & 2277 \\ \hline
			\textbf{Halbes Sudoku} & \multicolumn{1}{l|}{} & \multicolumn{1}{l|}{} & \multicolumn{1}{l|}{} & \multicolumn{1}{l|}{} \\ \hline
			javacc  & 472 & 483 & 580 & 489 \\ \hline
			kern, slim   & 1657 & 1606 & 1628 & 1714 \\ \hline
			kern, full   & 3580 & 3473 & 3557 & 3741 \\ \hline
			instaparse, slim   & 5151 & 4708 & 5085 & 4941 \\ \hline
			\textbf{Sudoku} & \multicolumn{1}{l|}{} & \multicolumn{1}{l|}{} & \multicolumn{1}{l|}{} & \multicolumn{1}{l|}{} \\ \hline
			javacc  & 695 & 1182 & 1028 & 1044 \\ \hline
			kern, slim   & 3051 & 2941 & 3014 & 3198 \\ \hline
			kern, full   & 6977 & 6631 & 6777 & 7279 \\ \hline
			instaparse, slim   & 19072 & 18782 & 18530 & 18439 \\ \hline
		\end{tabular}
	}
		\label{Messergebnisse}
		\caption{Ergebnisse der Messreihen, alle Angaben in Millisekunden}
\end{table}


\end{document}
